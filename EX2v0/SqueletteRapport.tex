% debut d'un fichier latex standard
\documentclass[12pt,twoside]{article}
% pour l'inclusion de figures en eps,pdf,jpg,....
\usepackage{graphicx}
% quelques symboles mathematiques en plus
\usepackage{amsmath}
% le tout en langue francaise
\usepackage[french]{babel}
% on peut ecrire directement les characteres avec l'accent
% a utiliser sur Linux/Windows
\usepackage[utf8]{inputenc}
% a utiliser sur le Mac
%\usepackage[applemac]{inputenc}
% pour l'inclusion de links dans le document (pdflatex)
\usepackage[colorlinks,bookmarks=false,linkcolor=blue,urlcolor=blue]{hyperref}
%
% quelques abreviations utiles
\def \be {\begin{equation}}
\def \ee {\end{equation}}
\def \dd  {{\rm d}}
%
\newcommand{\mail}[1]{{\href{mailto:#1}{#1}}}
\newcommand{\ftplink}[1]{{\href{ftp://#1}{#1}}}
%
% le document commence ici
\begin{document}
% le titre et l'auteur
\title{Physique numérique I : exercice 2}
\date{\today}
\author{Victor Despland, Timothée Dao\\{\small \mail{victor.despland@epfl.ch}, \mail{timothee.dao@epfl.ch}}}
\maketitle
\tableofcontents
\section{Introduction}

Un proton de masse $m = 1.6726  \times 10^{-27 } \rm kg$, de charge $ = 1.6022 \times 10^{-19} \rm C$, avec une position initiale $(x_0, y_0 )$ et une vitesse initiale $\textbf{v}_0 = (v_{x0} , v_{y0} )$, est plongé dans un champ électrique uniforme $\textbf{E} = E\hat{y}$ et un champ magnétique uniforme $\textbf{B} = B\hat{z} $ . Il est alors soumis à la force de Lorentz $\textbf{F} = q(\textbf{E} + \textbf{v} \times\textbf{ B})$.

\section{Calculs analytiques}

De la 2ème loi de Newton, $\vec{F}=m\vec{a}$, on tire :
\be
\frac{\dd }{\dd t} 
\left( \begin{array}{c} v_x \\ v_y \\ x \\ y \end{array} \right)
=
\left( \begin{array}{c}
   \frac{q}{m} v_y B\\ \frac{q}{m} (E-v_x B) \\ 
   v_x \\ v_y
\end{array} \right)
\ee
Dans le cas où $E=0$, il s'agit de résoudre dans un premier temps l'équation differentielle suivante 
\be
\frac{\dd }{\dd t} 
\left( \begin{array}{c} v_x \\ v_y \end{array} \right)
=
\lambda
\left( \begin{matrix}
 0 & 1 \\
 -1 & 0
\end{matrix} \right)
\left( \begin{array}{c}
   v_{x_0} \\ v_{y_0}
\end{array} \right),
\textbf{v}(0) = \textbf{v}_0
\ee
avec $\lambda=\frac{qB_0}{m}$.
La solution est donnée par 
\be
\textbf{v}(t)=\exp(At) \textbf{v}_0=
\left( \begin{array}{c}
   v_{x_0}  \cos (\lambda t) + v_{y_0} \sin (\lambda t) \\
   -v_{x_0} \sin (\lambda t) + v_{y_0} \cos (\lambda t)
\end{array} \right)
\ee
On résout finalemement 
\be
\frac{\dd }{\dd t} 
\left( \begin{array}{c} x \\ y \end{array} \right)
=
\left( \begin{array}{c} v_x \\ v_y \end{array} \right) , 
\left( \begin{array}{c} x(0) \\ y(0) \end{array} \right)
=
\left( \begin{array}{c}  x_0 \\ y_0 \end{array} \right) 
\ee
La solution est donnée par 
\be
\left( \begin{array}{c} x \\ y \end{array} \right) (t)
=
\frac{1}{\lambda} \left( \begin{array}{c}
   v_{x_0}  \sin (\lambda t) - v_{y_0} \cos (\lambda t) + v_{y_0} + \lambda x_0 \\
   v_{x_0} \cos (\lambda t) + v_{y_0} \sin (\lambda t) - v_{x_0} + \lambda y_0
\end{array} \right) 
\ee
\subsection{ABC}

\subsection{Figures} \label{sec:figures}

% une figure 'flottante', c'est a dire que c'est Latex qui va vous placer
% la figure la ou il lui semble bon.
\begin{figure}
% decommenter la ligne suivante quand vous avez prepare les fichiers
% eps ou pdf
%\centerline{\includegraphics[width=0.9\linewidth,angle=0]{Plot}}
% la legende est dans caption
\caption{\em
 la l\'egende.
 \label{fig:Plot}
}
\end{figure}
\begin{thebibliography}{99}
\bibitem{ref1} 
 Une premi\`ere r\'ef\'erence.
 \end{thebibliography}
\end{document}
